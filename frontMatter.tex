%% \documentclass[preprint,12pt,3p]{elsarticle}
%% Use the option review to obtain double line spacing
\documentclass[preprint,review,12pt]{elsarticle}

\usepackage{amssymb}
\usepackage{amsmath}
\usepackage{graphicx}
\usepackage[retainorgcmds]{IEEEtrantools}
\usepackage[colorlinks]{hyperref}
\hypersetup{
  citecolor = {blue}
}
\usepackage{siunitx}
\usepackage{pdflscape}
% disable labelindent
\let\labelindent\relax
\usepackage{enumitem}
\usepackage[normalem]{ulem}
\useunder{\uline}{\ul}{}
\usepackage{setspace} % \doublespacing
\usepackage[utf8]{inputenc}
\usepackage[linesnumbered, ruled, vlined]{algorithm2e}
\usepackage{nicefrac}
\usepackage{xspace}
\usepackage{float}
\usepackage{caption}
\usepackage{subcaption}
\usepackage{rotating, lscape, longtable, tabu, booktabs, siunitx, multirow}
\usepackage[capitalise, noabbrev]{cleveref}
\usepackage{verbatim}
\usepackage{mathtools}

% tikz libraries
\usepackage{tkz-graph}

\usetikzlibrary{arrows,arrows.meta,shapes,decorations.pathmorphing, decorations.markings, positioning}
\let\svtikzpicture\tikzpicture
\def\tikzpicture{\noindent\svtikzpicture}

% personal definitions
\newcommand{\uset}[1]{\ifmmode\left\{\,#1\,\right\}\else\{\,#1\,\}\fi}
\newcommand{\ulst}[1]{\ifmmode\left[\,#1\,\right]\else[\,#1\,]\fi}
\newcommand{\upar}[1]{\ifmmode\left(\,#1\,\right)\else(\,#1\,)\fi}
\newcommand{\uioc}[1]{\ifmmode\left(\,#1\,\right]\else(\,#1\,]\fi}
\newcommand{\uico}[1]{\ifmmode\left[\,#1\,\right)\else[\,#1\,)\fi}

\newcommand{\ie}{i.\,e.,\xspace}
\newcommand{\eg}{e.\,g.,\xspace}
\newcommand{\false}{\texttt{false}}
\newcommand{\true}{\texttt{true}}

% Theorem Styles
\newtheorem{theorem}{Theorem}
\newtheorem{lemma}[theorem]{Lemma}
\newtheorem{proposition}[theorem]{Proposition}
\newtheorem{corollary}[theorem]{Corollary}
\newtheorem{conjecture}[theorem]{Conjecture}
\newenvironment{proof}{\noindent\textbf{Proof}.}{\hfill$\square$}
% Definition Styles
\newtheorem{definition}{Definition}
\newtheorem{example}{Example}
\newtheorem{remark}{Remark}

\journal{Journal name}

\begin{document}

\begin{frontmatter}


\title{Article title}


\author[unifal]{Iago A. Carvalho\corref{cor1}}
\ead{iago.carvalho@unifal-mg.edu.br}
\ead[url]{iagoac.github.io/}

\author[label2]{Homer J. Simpson}
\ead{homer@thesimpsons.com}

\address[unifal]{Department of Computer Science, Universidade Federal de Alfenas}

\cortext[cor1]{I am corresponding author}

\address[label2]{Twenty Century Fox, USA}

\begin{abstract}
There will be an abstract here.
\end{abstract}

\begin{keyword}
Keyword 1 \sep Keyword 2 \sep Keyword 3 \sep Keyword 4 \sep Keyword 5
\end{keyword}

\end{frontmatter}


% \linenumbers

\section{Introduction} \label{sec:intro}

% Contributions of the paper
In this paper, we first...

% Organization of the paper
The remainder of this paper is organized as follows.
Section~\ref{sec:problem} formally defines our problem.
Section~\ref{sec:related} reviews the related works.
Section~\ref{sec:algorithms} presents the algorithms, which are evaluated on Section~\ref{sec:experiments}. Finally, Section~\ref{sec:conclusions} draws the concluding remarks of our paper.

\section{The problem} \label{sec:problem}


\section{Related works} \label{sec:related}


\section{Algorithms} \label{sec:algorithms}

\subsection{Algorithm 1}

\subsection{Algorithm 2}

\subsection{Algorithm n}


\section{Computational experiments} \label{sec:experiments}

% Computational envoirnment description
The computational experiments have been done on a single core of an Intel
with x.x Ghz clock and x Gb of RAM,
unning underthe  operating  system  Linux  Ubuntu.
We used the ILOG  CPLEX  solver  version  12.8  with default parameters setting. 
The algorithms were implemented in C{}\verb!++! along with the ILOG Concert Technology and compiled with the GNU g{}\verb!++! 8.2.0. 
The running time of all algorithms has been set to 7200 seconds.


\subsection{Instance's description} \label{subsec:instances}


\subsection{Expriment 1}

\subsection{Experiment 2}

\subsection{Experiment m}


\section{Conclusions} \label{sec:conclusions}


\section*{Acknowledgments}
This study was financed in part by the \emph{Coordenação de Aperfeiçoamento de Pessoal de Nível Superior - Brazil} (CAPES) - Finance Code 001, the \emph{Conselho Nacional de Desenvolvimento Científico e Tecnológico - Brazil} (CNPq), and the \emph{Fundação de Amparo à Pesquisa do Estado de Minas Gerais - Brazil} (FAPEMIG).

\bibliographystyle{elsarticle-num}
% \bibliographystyle{elsarticle-harv}
% \bibliographystyle{elsarticle-num-names}
% \bibliographystyle{model1a-num-names}
% \bibliographystyle{model1b-num-names}
% \bibliographystyle{model1c-num-names}
% \bibliographystyle{model1-num-names}
% \bibliographystyle{model2-names}
% \bibliographystyle{model3a-num-names}
% \bibliographystyle{model3-num-names}
% \bibliographystyle{model4-names}
% \bibliographystyle{model5-names}
% \bibliographystyle{model6-num-names}

\bibliography{bibsample}

\end{document}

%%
%% End of file `elsarticle-template-num.tex'.
